\capitulo{2}{Objetivos del proyecto}

A continuación se describen los objetivos que se pretenden alcanzar con el desarrollo de este proyecto:
\begin{itemize}
\tightlist
	\item Desarrollar un videojuego de género \textit{shooter}, en concreto del estilo \textit{battle royale}, para ordenador como plataforma de juego, compatible con Windows y Linux y pensado para controlarse con teclado y ratón. Este objetivo puede subdividirse a su vez en objetivos más concretos:
    	\begin{itemize}
    	\tightlist
    	\item Desarrollar una buena \textbf{jugabilidad} y mecánicas interesantes para que el producto resulte entretenido y divertido para los usuarios.
    	\item Hacer que el producto sea \textbf{accesible} para un gran espectro de usuarios, es decir, que aprendan las mecánicas y acciones que ofrece el videojuego de forma intuitiva y con una curva de aprendizaje sencilla y no frustrante.
    	\item Crear de manera \textbf{autónom}a la mayoría de componentes del videojuego, incluyendo scripts, modelos, texturas, efectos, elementos de interfaces y sonidos.
    	\item Hacer que el videojuego sea \textbf{óptimo} en términos de rendimiento.
    	\item Utilizar \textbf{patrones, algoritmos y estructuras} de programación convenientes en cada caso.
    	\item Hacer que el proyecto sea fácilmente \textbf{extensible} para incluir mecánicas, objetos, mejoras o modificaciones de manera sencilla e intuitiva.
    	\end{itemize}
	\item Aprender el funcionamiento y las características fundamentales de Unity, así como del lenguaje de programación que utiliza (C\#)
	\item Aplicar la metodología \textit{Kanban} durante el desarrollo del producto para la gestión de las tareas.
	\item Integrar Git y Github como sistema de control de versiones.
	\item Asentar las bases de conocimiento personal para poder realizar otros proyectos similares en el futuro o continuar el desarrollo y mejora del mismo.
\end{itemize}